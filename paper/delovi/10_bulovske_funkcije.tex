\section{Bulovske funkcije}
\label{sec:BulovskeFunkcije}

\emph{Bulovske funkcije} su funkcije koje primaju bulovske argumente i vra\'c{}aju bulovsku vrednost. Bulovske vrednosti mogu biti \emph{true} ili \emph{false}. U nastavku \'c{}emo sa $1$ ozna\v{c}avati \emph{true}, a sa $0$ \emph{false}, \v{s}to je uobi\v{c}ajena konvencija.

Za bulovsku funkciju sa $n$ bulovskih argumenata, postoji $2^{n}$ mogu\'c{}ih ulaza. Po\v{s}to je povratna vrednost takodje bulovskog tipa, zaklju\v{c}ujemo da postoji $2^{2^{n}}$ razli\v{c}itih bulovskih funkcija sa $n$ argumenata, \v{s}to se vidi iz slede\'c{}e jednakosti:

\[ \underbrace{2*2*2* \dotsb *2}_{\text{$2^{n}$}} = 2^{2^{n}} \]

Funkcije koje primaju neozna\v{c}eni broj u opsegu $[0,2^{n}-1]$ se mogu zameniti sa $n$ bulovskih funkcija sa $n$ argumenata. Kao primer, neka je data funkcija $F$ koja prima i vra\'c{}a neozna\v{c}eni ceo broj. Zamenjujemo funkciju $F$ bulovskim funkcijama $f_i$, gde $i = 0, 1, \dots , n-1$. Argumenti funkcije $f_{i}$ su $n$ binarnih cifara broja, dok je povratna vrednost $f_{i}$ vrednost $i$-te binarne cifre rezultata funkcije $F$. Drugim re\v{c}ima, svaka od funkcija $f_{i}$ ra\v{c}una jednu cifru rezultata. Kao konkretan primer, s obzirom da su neozna\v{c}eni brojevi u ra\v{c}unarima zauzimaju obi\v{c}no $32$ bita, funkciju:

\begin{lstlisting}[language=C++]
    unsigned F(unsigned n);
\end{lstlisting}

\noindent mo\v{z}emo zameniti sa $32$ bulovske funkcije:

\begin{lstlisting}[language=C, emph={bool}]
    bool f0(bool n0, bool n1, bool n2, ... , bool n31);
    bool f1(bool n0, bool n1, bool n2, ... , bool n31);
    ...
    bool f31(bool n0, bool n1, bool n2, ... , bool n31);
\end{lstlisting}

Naravno, moramo se uveriti da su cifre rezultata zaista jednake izlazima bulovskih funkcija. Jedan na\v{c}in za verifikaciju rezultata je primena obe tehnike nad svim mogu\'c{}im ulazima i poredjenje dobijenih rezultata. Medjutim, \v{c}ak i za ovako jednostavne funkcije re\v{c} je o oko 4 milijarde ($2^{32}$) mogu\'c{}ih ulaza. Drugi na\v{c}in je da se ove funkcije prika\v{z}u putem nekih struktura podataka, i da se funkcije porede tako \v{s}to se porede njihove reprezentacije preko tih strukture. Drugim re\v{c}ima, posmatramo funkcije kao podatke. U poglavljima koji slede \'c{}e biti vi\v{s}e re\v{c}i o strukturama podataka koje se koriste za predstavljanje bulovskih funkcija. Takodje, u narednim poglavljima pod terminom \emph{funkcija} \'c{}emo podrazumevati bulovske funkcije, ukoliko to nije druga\v{c}ije nazna\v{c}eno.
