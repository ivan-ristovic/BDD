\emph{Binary Decision Diagrams} (u daljem tekstu \emph{BDD}) i njihova pobolj\v{s}anja su strukture podataka za reprezentaciju bulovskih funkcija. Iako u osnovi sli\v{c}ni binarnim drvetima, re\v{s}avaju problem velikog broja \v{c}vorova u drvetu uklanjaju\'c{}i redundantne grane (za bulovsku funkciju sa $n$ argumenata, broj mogu\'c{}ih puteva u binarnom drvetu od korena do lista je $2^{n}$, dok je broj \v{c}vorova znatno ve\'c{}i). U ovom radu \'c{}emo detaljnije opisati intuiciju iza BDD struktura, na\v{c}ine za konstrukciju BDD, kao i \emph{ROBDD} - redukovana i uredjena BDD. Ovaj rad \'c{}e pratiti implementacija BDD u jeziku \emph{C++}, uz prate\'c{}e delove koda na nekim mestima.

% todo remove, need atleast one reference to it so the compilation can work
\cite{BDD}
