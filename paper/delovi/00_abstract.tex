\emph{Binarni dijagrami odlu\v{c}ivanja} (u daljem tekstu \emph{BDD}) i njihova pobolj\v{s}anja su strukture podataka za reprezentaciju bulovskih funkcija. Iako u osnovi sli\v{c}ni binarnim drvetima, re\v{s}avaju problem velikog broja \v{c}vorova u drvetu uklanjaju\'c{}i redundantne grane (za bulovsku funkciju sa $n$ argumenata, broj mogu\'c{}ih puteva u binarnom drvetu od korena do lista je $2^{n}$, dok je broj \v{c}vorova znatno ve\'c{}i). U ovom radu \'c{}emo detaljnije opisati ideju na kojoj su zasnovani BDD, na\v{c}ine za njihovu konstrukciju, kao i razne varijante BDD - \emph{ROBDD}, \emph{FBDD} i \emph{ZDD}. Ovaj rad \'c{}e pratiti implementacija ROBDD u jeziku \emph{C++}, uz prate\'c{}e delove koda na nekim mestima.
