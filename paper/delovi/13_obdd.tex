\section{Redukovani uredjeni binarni dijagrami odlu\v{c}ivanja (ROBDD)}
\label{sec:OBDD}

Uredjeni binarni dijagrami odlu\v{c}ivanja (engl. \emph{Ordered Binary Decision Diagrams}, u daljem tekstu \emph{OBDD}) su BDD u kojima se promenljive uvek testiraju u istom redosledu. OBDD se naziva \emph{redukovano}(engl. \emph{Reduced Ordered Binary Decision Diagrams} \cite{ROBDD}, u daljem tekstu \emph{ROBDD}) ukoliko ima slede\'c{}e osobine:
\begin{itemize}
    \item Neredundantnost - Visoko i nisko podstablo svakog \v{c}vora je razli\v{c}ito.
    \item Jedinstvenost - Ne postoje dva razli\v{c}ita \v{c}vora koja testiraju istu promenljivu, a \v{c}ija su podstabla ista.
\end{itemize}

Kao \v{s}to je pomenuto u poglavlju \ref{sec:BinarnaDrvetaOdlucivanja}, binarna drveta odlu\v{c}ivanja poseduju osobinu kanoni\v{c}nosti. Medjutim, kako postoje razli\v{c}ita BDD koja odgovaraju jednom binarnom drvetu odlu\v{c}ivanja, jasno je da BDD nemaju ovu osobinu. Uprkos tome, zbog svojih lepih i jasno definisanih osobina, ROBDD je takodje kanoni\v{c}no. Pro\v{s}irivanjem definicije kanoni\v{c}nosti na ROBDD, dobijamo tvrdjenje: kanoni\v{c}nost ROBDD zna\v{c}i da za fiksan redosled promenljivih, svaka bulovska funkcija ima jedinstvenu reprezentaciju u vidu ROBDD. Ovo zna\v{c}i da mo\v{z}emo uporediti bulovske funkcije konstrukcijom njihovih ROBDD, a ukoliko su ona jednaka, to garantuje njihovu ekvivalentnost.

ROBDD ima najvi\v{s}e dva lista: 0 i 1. U nastavku \'c{}emo ponekad crtati oba, kako bi se smanjila kompleksnost njhovog grafi\v{c}kog prikaza.



\subsection{Konstrukcija ROBDD}
\label{subsec:ROBDDConstruction}

Lalala konstrukcija Lalala

