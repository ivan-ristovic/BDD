\section{Binarni dijagrami odlu\v{c}ivanja sa potisnutim nulama (ZDD)}
\label{sec:ZDD}

Binarni dijagrami odlu\v{c}ivanja sa potisnutim nulama (engl. \emph{Zero-suppressed binary decision diagrams}, u daljem tekstu \emph{ZDD}) \cite{ZDD} predstavljaju vrstu BDD zasnovanih na specijalnom pravilu redukcije. ZDD su posebno efikasni pri re\v{s}avanju problema teorije skupova. Naime, za razliku od BDD koji su dobri za reprezentaciju funkcija, ZDD su dobri za reprezentaciju \emph{pokrivanja} - s obzirom da postoji korespodencija izmedju re\v{s}enja bulovskih funkcija i familija skupova.

\begin{defn}
    \emph{ZDD} je bilo koji uredjeni BDD sa dodatnim pravilom da \emph{pozitivna grana} svakog unutra\v{s}njeg \v{c}vora ne sme da bude povezana sa 0-\v{c}vorom. Ukoliko ovo pravilo nije ispunjeno ili ukoliko postoje redundantni \v{c}vorovi, rezultuju\'c{}i ZDD se naziva \emph{neredukovani ZDD}.
\end{defn}

Kao i BDD, ZDD koristi dva pravila za redukciju - \emph{pravilo spajanja} i \emph{pravilo eliminacije}. Pravilo spajanja izomorfnih pod-grafova je isto kao i kod BDD. Medjutim, postoji razlika u pravilu eliminacije. Definisa\'c{}emo formalno pomenuta pravila.

\begin{defn}
    \emph{Pravilo spajanja za ZDD}: Ako unutra\v{s}nji \v{c}vorovi $v_{1}$ i $v_{2}$ testiraju istu promenljivu i njihove pozitivne i negativne grane vode ka istim \v{c}vorovima, tada elimini\v{s}emo jedan od \v{c}vorova $u$ ili $v$ i sve ulazne grane u eliminisani \v{c}vor preusmeravamo ka preostalom \v{c}voru.
\end{defn}

\begin{defn}
    \emph{Pravilo eliminacije za ZDD}: Ako \emph{pozitivna grana} \v{c}vora $v$ spaja \v{c}vor $v$ sa \emph{0-\v{c}vorom} $v_{0}$ (konstantni \v{c}vor sa vredno\v{s}\'c{}u $0$), onda elimini\v{s}emo \v{c}vor $v$ i preusmeravamo sve ulazne grane u $v$ ka $v_{0}$.
\end{defn}

Pomenuli smo kako postoji veza izmedju bulovskih funkcija i familija skupova. Pojasnimo malo bolje tu vezu. Svako re\v{s}enje bulovske funkcije korespondira sa odredjenim podskupom prostora valuacija. Iako se familije skupova mogu kodirati kao bulovske funkcije, nekada ih je bolje posmatrati u svom osnovnom obliku. U tome se ogleda vrednost ZDD.
